\documentclass [11pt,a4paper,french]{report}

%====================================================
%%			    	    Package                        %%
%====================================================
\usepackage[usenames,dvipsnames]{xcolor}
\usepackage{tcolorbox}
\usepackage[utf8x]{inputenc} % Encodage du fichier tex
\usepackage[T1]{fontenc} % Pour avoir les lettres accentuees font encoding
\usepackage{lmodern} % Latin modern - caracteres diacritiques fran\c cais
\usepackage[english,francais]{babel}

\usepackage{lipsum} % dummy text
\usepackage{caption}
\usepackage{amsmath,latexsym,amsfonts,amssymb}
\usepackage{graphicx} % support the \includegraphics command and options
\usepackage{subfig} % make it possible to include more than one captioned figure/table in a single float
\usepackage[toc, title]{appendix}
\renewcommand\appendixtocname{Annexes}
\usepackage{color}% color text
%\usepackage[parfill]{parskip} % Activate to begin paragraphs with an empty line rather than an indent 
\usepackage{setspace} % augmentation de l'espace interligne
\setstretch{1.2}	 % inter-line
\usepackage{float}
\usepackage{amsmath}
\usepackage{amssymb}

\usepackage[hidelinks]{hyperref} % Rend les liens cliquables
\usepackage[toc, style=listgroup]{glossaries} % Ajout d'un glossaire. Celui ci apparait dans la table des matieres
%\glossarystyle{altlistgroup}
\loadglsentries{sections/Glossaire} % load the glossary file
\makeglossaries % Add glossary
\usepackage{url}
\usepackage{enumerate} 
\usepackage{booktabs} % for much better looking tables
\usepackage{array} % for better arrays (eg matrices) in maths
\usepackage{paralist} % very flexible & customisable lists (eg. enumerate/itemize, etc.)
\usepackage{verbatim} % adds environment for commenting out blocks of text & for better verbatim
\usepackage{enumitem} % to modifiy the chip in enumerate environment


%%% HEADERS & FOOTERS
\usepackage{fancyhdr}


\usepackage{pdfpages}


%%% SECTION TITLE APPEARANCE
\usepackage{sectsty}
\allsectionsfont{\sffamily\mdseries\upshape} % (See the fntguide.pdf for font help)
% (This matches ConTeXt defaults)

%%% ToC (table of contents) APPEARANCE
\usepackage[nottoc,notlof,notlot]{tocbibind} % Put the bibliography in the ToC
\usepackage[titles,subfigure]{tocloft} % Alter the style of the Table of Contents

\usepackage{amsfonts}
\usepackage{amssymb}
\usepackage{amsmath}
\usepackage{multirow}
\usepackage{amsthm}
\usepackage{dsfont}
\usepackage{calrsfs}
%possibilite de mettre en place des pages en mode paysage
\usepackage{lscape}
\usepackage{pdflscape}
\usepackage{everypage}
%\usepackage[landscape]{geometry}
%\usepackage{rotating}
%--------------------------------- Page format

\usepackage[left=3.5cm,top=3cm,bottom=2cm,right=2.5cm]{geometry}
\usepackage{tikzpagenodes}



%--------------------------------- figure and caption format

\usepackage[font=small,skip=0pt]{caption}


\usepackage{tikz}
\usetikzlibrary{shadows,calc}

% code adapted from https://tex.stackexchange.com/a/11483/3954

% some parameters for customization
\def\shadowshift{3pt,-3pt}
\def\shadowradius{6pt}

\colorlet{innercolor}{black!60}
\colorlet{outercolor}{gray!05}

\newcommand{\source}[1]{\caption*{Source {#1}} }

% this draws a shadow under a rectangle node
\newcommand\drawshadow[1]{
	\begin{pgfonlayer}{shadow}
		\shade[outercolor,inner color=innercolor,outer color=outercolor] ($(#1.south west)+(\shadowshift)+(\shadowradius/2,\shadowradius/2)$) circle (\shadowradius);
		\shade[outercolor,inner color=innercolor,outer color=outercolor] ($(#1.north west)+(\shadowshift)+(\shadowradius/2,-\shadowradius/2)$) circle (\shadowradius);
		\shade[outercolor,inner color=innercolor,outer color=outercolor] ($(#1.south east)+(\shadowshift)+(-\shadowradius/2,\shadowradius/2)$) circle (\shadowradius);
		\shade[outercolor,inner color=innercolor,outer color=outercolor] ($(#1.north east)+(\shadowshift)+(-\shadowradius/2,-\shadowradius/2)$) circle (\shadowradius);
		\shade[top color=innercolor,bottom color=outercolor] ($(#1.south west)+(\shadowshift)+(\shadowradius/2,-\shadowradius/2)$) rectangle ($(#1.south east)+(\shadowshift)+(-\shadowradius/2,\shadowradius/2)$);
		\shade[left color=innercolor,right color=outercolor] ($(#1.south east)+(\shadowshift)+(-\shadowradius/2,\shadowradius/2)$) rectangle ($(#1.north east)+(\shadowshift)+(\shadowradius/2,-\shadowradius/2)$);
		\shade[bottom color=innercolor,top color=outercolor] ($(#1.north west)+(\shadowshift)+(\shadowradius/2,-\shadowradius/2)$) rectangle ($(#1.north east)+(\shadowshift)+(-\shadowradius/2,\shadowradius/2)$);
		\shade[outercolor,right color=innercolor,left color=outercolor] ($(#1.south west)+(\shadowshift)+(-\shadowradius/2,\shadowradius/2)$) rectangle ($(#1.north west)+(\shadowshift)+(\shadowradius/2,-\shadowradius/2)$);
		\filldraw ($(#1.south west)+(\shadowshift)+(\shadowradius/2,\shadowradius/2)$) rectangle ($(#1.north east)+(\shadowshift)-(\shadowradius/2,\shadowradius/2)$);
	\end{pgfonlayer}
}

% create a shadow layer, so that we don't need to worry about overdrawing other things
\pgfdeclarelayer{shadow} 
\pgfsetlayers{shadow,main}


\newcommand\shadowimage[2][]{%
	\begin{tikzpicture}
	\node[anchor=south west,inner sep=0] (image) at (0,0) {\includegraphics[#1]{#2}};
	\drawshadow{image}
	\end{tikzpicture}}


\setlength{\textfloatsep}{-10pt plus 1.0pt minus 2.0pt}


%--------------------------------- Bash format code 
\usepackage{listings}
\tcbuselibrary{listings,skins} 


\lstdefinestyle{bash}{
	language=bash,
	basicstyle=\ttfamily\color{white},
	breaklines=true
}

\newtcblisting{bash}{
	enhanced,                             %%% needed for shadow
	arc=2.5mm,
	top=0mm,
	bottom=0mm,
	left=2mm,
	right=2mm,
	boxrule=0pt,
	colback=black,
	%shadow={5mm}{-3mm}{0mm}{fill=black!50!white,
	%	opacity=0.5},             %%% here for shadow  and adjust as you like
	listing only,
	listing options={style=bash},
	%hbox
}
%--------------------------------- info bulle 


\usepackage[tikz]{bclogo}
\usepackage{etoolbox}
%regle la distance entre les figure et le texte , padding haut et bas d'une figure. 
\apptocmd{\thebibliography}{\raggedright}{}{}


\setlength{\intextsep}{20pt plus 10.0pt minus 5.0pt}


%Mise en place de la numerotation en bas de pages dans le cadre du mode paysage cf annexe

\newcommand{\Lpagenumber}{\ifdim\textwidth=\linewidth\else\bgroup
	\dimendef\margin=0
	\ifodd\value{page}\margin=\oddsidemargin
	\else\margin=\evensidemargin
	\fi
	\raisebox{\dimexpr -\topmargin-\headheight-\headsep-0.5\linewidth}[0pt][0pt]{%
		\rlap{\hspace{\dimexpr \margin+\textheight+\footskip}%
			\llap{\rotatebox{90}{\thepage}}}}%
	\egroup\fi}
\AddEverypageHook{\Lpagenumber}%

\newcommand\fnote[1]{\captionsetup{font=small}\caption*{#1}}


%exemple de macro pour le mode bash

%\begin{figure}[h!]
%	\caption{NOM DU CAPTION}
%	\label{NOM DU LABEL}
%	\begin{bash}
%		root@zeewa:~#
%	\end{bash}
%\end{figure}
%

%exemple de macro pour le mode commentaire 
%\begin{bclogo}[logo=\bctrombone, couleurBarre=yellow, couleur = blue!20, arrondi=0.5,marge=10, noborder=true,ombre=true, couleurOmbre=black!30,blur]{NOM}
%\end{bclogo}


