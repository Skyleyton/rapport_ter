\chapter{Écosystème  étudiant}
\label{chap:écosysteme}

Selon la définition anglaise de Wikipédia, un écosystème est : 

\begin{bclogo}[logo=\bctrombone, couleurBarre=yellow, couleur = blue!20, arrondi=0.5,marge=10, noborder=true,ombre=true, couleurOmbre=black!30,blur]{Définition écosystème informatique}
\lipsum[2]
\end{bclogo}

\lipsum[4]
\newpage
%-------------------------------
\section{HTML5}
%-------------------------------
Utilisation de technologie web~~ \raisebox{-.25\height}{\shadowimage[width=15mm]{html5_logo.png}}\\\\
\lipsum[4]


%-------------------------------
\section{Divers logiciels de travail}
%-------------------------------

\begin{bclogo}[logo=\bcattention, couleurBarre=red, couleur = blue!20, arrondi=0.5,marge=10, noborder=true,ombre=true, couleurOmbre=black!30,blur]{Compléments d'informations sur les outils mis à disposition}
\lipsum[4] 
\end{bclogo}

%-------------------------------
\section{Définition de  Mise En Production - MEP }
%-------------------------------
%https://fr.wikipedia.org/wiki/Gestion_des_mises_en_production

D'après la définition de Wikipédia \cite{Gestion_mise_en_production} : 

\begin{bclogo}[logo=\bctrombone, couleurBarre=yellow, couleur = blue!20, arrondi=0.5,marge=10, noborder=true,ombre=true, couleurOmbre=black!30,blur]{Mise en production}
\lipsum[4]
\end{bclogo}
% comme vu au chapitre \ref{production}


\subsection{Connexion au serveur}


\lipsum[4]

\begin{figure}[H]
\begin{bash}
Last login: Mon Feb 20 21:55:34 on ttys000
seb@mbpsebastien:~# ssh-copy-id root@toto.com -p 142	
\end{bash}
\caption{Mise en place d'une clé de sécurité sur un serveur.}
\label{fig:keys}
\end{figure}
