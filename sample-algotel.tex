\documentclass[%
% if you want to use pdflatex, uncomment the following line
%pdftex
% if you have difficulties with fonts uncomment the following line
%notimes
]{algotel}

\usepackage{lmodern} % Latin modern - police vectorielle
\usepackage[english, french]{babel} % typographie francaise
\usepackage[utf8]{inputenc} % Encodage du fichier tex
\usepackage[T1]{fontenc} % Affichage des caractères accentués

\author{Jens Gustedt\addressmark{1}\thanks{I am not supported.}{\ }
  \and Somebody Who\addressmark{2}\thanks{But he is!}{\ }
  \and Some Dummy\addressmark{1}}

\title[Formatting a submission for AlgoTel]{How to format a submission for AlgoTel\\  with the conference's own \LaTeXe-style}

\address{\addressmark{1}INRIA \& LORIA, campus scientifique, BP 239, F-54506 Vand{\oe}uvre lès Nancy, France \\
  \addressmark{2}Blue-box, Maribor, Slovenia}


\keywords{some well classifying words, \texttt{mandatory!}}

\begin{document}
\maketitle

\begin{abstract}
Un résumé rapide et en fran\c{c}ais de la contribution. 
\end{abstract}

\section{Introduction}
\label{sec:in}
The \LaTeXe-style for AlgoTel  is  derived straight forward from  the
DMTCS style which is also derived from the usual \texttt{article.sty}!
Its main purpose is to  ensure a common layout  policy of all articles
in AlgoTel and   to provide  editors,  referees and  readers  with the
necessary information.  If  you think  you   need an introduction   to
\LaTeXe or search  for  pointers to other   literature on  that,  you
should consider  reference~\cite{oetiker99:_not_so_short_introd_latex}
given at the end.

\bigskip


\section{Providing Information for the first Page}
\label{sec:first}


We need the following type of information:

\begin{itemize}
\item The name(s) of the author(s), provided by the \verb!\author!
  command.

  This is exactly the same as for standard \LaTeXe. Please refer to
  your \LaTeX\ book to see how this is usually done, or look at the
  examples given here in this file.
\item The title of the manuscript, provided by the \verb!\title!
  command.

  The title command may be given in two different forms. The first is 
\begin{verbatim}
\title{Your title goes here}
\end{verbatim}
  If done like that, the title that you give is used as running head
  for the odd numbered pages as well. If your title is too long such
  that it doesn't fit into the running head you should use the
  alternative form
\begin{verbatim}
\title[Formatting a submission for AlgoTEl]{How to format a submission 
for AlgoTEl with the journal's own \LaTeXe-style}
\end{verbatim}
  Here the string that inside the \verb![ ]! is used in the
  running head.
\item The address(es) of the authors, provided by the \verb!\address!
  command.
\item Some keywords that classify your work, provided by the \verb!\keywords!
  command. Be careful on the choice of these keywords, you are the
  author, you should know best what is adequate such that your
  article can be easily and correctly identified by search engines and 
  alike. Give it in the form
\begin{verbatim}
\keywords{first item, second, third}
\end{verbatim}
  So each ``\emph{key word}'' might consist of several words in the
  usual sense. To separate several key words use commas.

  These keywords must be the same as the ones that are given when you
  fill out the http-form for submission.
\item An abstract of you manuscript, provided by the \verb!abstract!
  environment. This should be no longer than a paragraph and concisely 
  reflect the main contributions of your work. 
  
  
\end{itemize}


\section{Hints for the manuscript itself}
\label{sec:hints}


\subsection{Numbering commands}
\label{sec:numbering}

Please use the standard conventions for all commands and environments
that provide a numbering such as theoremlike environments or
sections. In particular usual counting starts at $1$ and not at
$0$. 



\subsection{Proper Names}
\label{sec:names}
Please also be careful in the writing of personal names. Customs in
different countries are different! Be sure to use a standard
transcription of names that use a different alphabet than English, and 
also be sure to use the full capabilities of \LaTeXe for accentuated
character sets that are based on the Latin alphabet. Be sure to catch
the correct concept of ``last name'' in that language.


\subsection{Use a Spell Checker}
\label{sec:check}

It is considered as being very impolite to leave obvious spelling
errors in the manuscript before sending it out. Computers are made for 
these, \textbf{use them}.

You might either use the North American variant for spelling or the
British one, but please don't mix them in one paper. The same holds
for different possible spellings for the same word as for example
``\emph{acknowledg(e)ment}'' or ``\emph{formulae}'' versus
``\emph{formulas}''. \textbf{Be coherent}.

\subsection{Mathematics}
\label{sec:math}
\begin{itemize}
\item Running text must always constitute correct English phrases.
  
\item All complicated mathematical formulae should be given on
  separate lines and should not be spread out into the running text.
  You should use \LaTeXe environments that provide a numbering for
  such formulae such as \texttt{equation} or \texttt{eqnarray}. Such
  numbers ease the referee process very much, and after eventual
  publication easily allow readers to refer to in their own work.
  
  
\item The quantifiers ``$\exists$'' and ``$\forall$'' don't stand as
  abbreviations of the partial phrases ``\emph{there is}'' and
  ``\emph{for all}''. They are reserved for logical formulae as
  \emph{such}, that is for work that talks itself of logical
  formulae as a subject.
  
  
\item The equal sign ``$=$'' has different meanings in parts of the
  two communities that DMTCS addresses.
  \begin{enumerate}
  \item It might stand for mathematical identity that is discovered
    \emph{a posteriori}. As an example take the following phrase:
    \begin{center}
      \emph{An easy computation shows that $4!=24$}.
    \end{center}
  \item It might stand for a \emph{definition}, as in
    \begin{center}
      \emph{For convenience, set $0!=1$.}
    \end{center}
  \end{enumerate}
  For the later use of ``$=$'' Computer Scientist often tend to use
  ``$:=$''. Referees should be tolerant to these different customs.
\end{itemize}



\section{PDF Files}
\label{sec:pdf}
The style now supports an option \texttt{pdftex} to use in combination
with \texttt{pdflatex}. This is only experimental, but we hope that
\texttt{pdflatex} will become more stable in the near future. If you
are viewing this document in its pdf form you may see some of the
advantages this has: in particular pdf documents produced in that way
have included \emph{hyperlinks}. If you want to know more about these
features please refer to
\href{http://xxx.lanl.gov/hypertex/}{\texttt{http://xxx.lanl.gov/hypertex/}}.

If your installation doesn't support the package \texttt{hyperref},
you should switch of these features by giving the option
\texttt{nohyperref} in the \verb!\documentclass! declaration at the
beginning of your manuscript.  To give you the possibility to include
hyperlinks even if your local installation doesn't support this, we
provide the command \verb!\href{URL}{text}! in any case.


\section{Summary of Options}
\label{sec:options}
\begin{tabular}{|l|p{6cm}|}
\hline
option & description\\
\hline
pdftex & to be enabled when processing with pdflatex\\
notimes & switch selection of the \texttt{times} package off\\
\hline
\end{tabular}


\section{Summary of Relevant Commands}
\label{sec:commands}

\begin{tabular}{|l|p{6cm}|}
\hline
command & description\\
\hline
\verb!\address! & the affiliation and address of the authors\\
\hline
\verb!\keywords! & a comma separated list of keywords\\
\hline
\verb!\qed! & produces \qed\\
\hline
\verb!\acknowledgements! & \\
\hline
\end{tabular}


\nocite{*}
\bibliographystyle{alpha}
\bibliography{sample-dmtcs}
\label{sec:biblio}

\end{document}